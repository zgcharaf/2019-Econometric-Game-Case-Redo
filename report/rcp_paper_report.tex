\documentclass[11pt]{article}

\usepackage[margin=1in]{geometry}
\usepackage[T1]{fontenc}
\usepackage[utf8]{inputenc}
\usepackage{lmodern}
\usepackage{setspace}
\usepackage{booktabs}
\usepackage{longtable}
\usepackage{graphicx}
\usepackage{float}
\usepackage{amsmath}
\usepackage{siunitx}
\usepackage{hyperref}
\usepackage{xcolor}
\usepackage{caption}
\usepackage{subcaption}

\hypersetup{
  colorlinks=true,
  linkcolor=blue,
  urlcolor=blue,
  citecolor=blue
}

\title{Econometric Carbon Budget Study:\\
Pipeline Approach and Results Summary}
\author{Auto-generated manuscript draft}
\date{\today}

\newcommand{\optionalfigure}[3]{%
  \IfFileExists{#1}{%
    \begin{figure}[H]
      \centering
      \includegraphics[width=#2\textwidth]{#1}
      \caption{#3}
    \end{figure}
  }{%
    \begin{figure}[H]
      \centering
      \fbox{\parbox{0.9\textwidth}{\centering Missing figure: \texttt{#1}.\\
      Copy this file from your pipeline output folder into the same Overleaf project path.}}
      \caption{#3 (placeholder)}
    \end{figure}
  }%
}

\newcommand{\optionaltableinput}[2]{%
  \IfFileExists{#1}{%
    \begin{table}[H]
      \centering
      \caption{#2}
      \input{#1}
    \end{table}
  }{%
    \begin{table}[H]
      \centering
      \caption{#2 (placeholder)}
      \fbox{\parbox{0.9\textwidth}{\centering Missing table: \texttt{#1}.\\
      Copy this file from your pipeline output folder into the same Overleaf project path.}}
    \end{table}
  }%
}

\begin{document}
\maketitle
\onehalfspacing

\begin{abstract}
This document summarizes the modeling pipeline used for the econometric carbon-budget study,
including data assembly, model estimation, diagnostics, and RCP-based inversion analysis.
It is written so you can upload directly to Overleaf and immediately render a readable report.
If figures/tables are not yet copied into the Overleaf project, placeholder boxes are shown,
which makes missing assets explicit and easy to fix.
\end{abstract}

\section{Study Objective}
The objective is to recover admissible emissions trajectories implied by concentration pathways and
driver-adjusted sink dynamics, while preserving a transparent, reproducible workflow. The pipeline
combines Global Carbon Budget components with exogenous climate drivers, estimates reduced-form
sink equations, and performs forward inversion under alternative assumptions (linear vs saturation)
and climate scenarios (baseline, adverse, favorable).

\section{Pipeline Overview}
\subsection{Data and feature engineering}
The workflow loads the historical carbon budget (fossil, land-use, ocean sink, land sink,
atmospheric growth, and imbalance), merges annual driver statistics (ENSO, drought, temperature,
and burned area), and constructs standardized regressors and emissions transforms.

\subsection{Econometric core}
For a selected estimation window, sink and imbalance relationships are fit and diagnosed. Outputs include:
\begin{itemize}
    \item model metrics,
    \item coefficient tables,
    \item residual diagnostic tables,
    \item stationarity checks.
\end{itemize}

\subsection{RCP inversion module}
Given an RCP concentration path, annual atmospheric growth is implied via first differences.
The inversion then solves for admissible total emissions under:
\begin{enumerate}
    \item a linear sink response,
    \item a nonlinear saturation extension with curvature parameter $\kappa$.
\end{enumerate}
Scenario perturbations produce baseline, bad-climate, and good-climate trajectories.

\section{How to Reproduce Results}
Run the pipeline (adjust paths as needed):
\begin{verbatim}
python -m src.pipeline run \
  --start 1989 --end 2012 \
  --drivers "nino34_max_z,scpdsi_global_max_z,tau_global_mean_z" \
  --label "tau-window best" \
  --rcp_dat "data/raw/RCP3PD_MIDYR_CONC.DAT" \
  --rcp_start 2020 --rcp_end 2100
\end{verbatim}

Then copy assets from the chosen run folder:
\begin{verbatim}
outputs/runs/<timestamp>/figures/* -> figures/
outputs/runs/<timestamp>/tables/*  -> tables/
\end{verbatim}
inside your Overleaf project (same level as this \texttt{.tex} file).

\section{Main Figures}
\optionalfigure{figures/budget_components.png}{0.95}{Historical global carbon budget components.}
\optionalfigure{figures/budget_imbalance.png}{0.90}{Historical budget imbalance.}
\optionalfigure{figures/rcp_implied_gatm.png}{0.95}{Implied atmospheric growth from RCP concentration path.}
\optionalfigure{figures/rcp_emissions_all_specs.png}{0.98}{All RCP-implied emissions specifications (linear and saturation).}
\optionalfigure{figures/rcp_emissions_baseline_linear_vs_sat.png}{0.95}{Baseline scenario: linear vs saturation assumptions.}
\optionalfigure{figures/rcp_emissions_bad_linear_vs_sat.png}{0.95}{Bad-climate scenario: linear vs saturation assumptions.}
\optionalfigure{figures/rcp_emissions_good_linear_vs_sat.png}{0.95}{Good-climate scenario: linear vs saturation assumptions.}
\optionalfigure{figures/rcp_emissions_scenario_envelope.png}{0.95}{Scenario envelope (min/median/max admissible emissions).}
\optionalfigure{figures/fig_pred_actual_vs_fitted.png}{0.95}{Prediction diagnostics: actual vs fitted atmospheric growth with uncertainty band.}
\optionalfigure{figures/fig_pred_residuals_vs_fitted.png}{0.90}{Prediction diagnostics: residuals vs fitted with LOWESS trend.}
\optionalfigure{figures/fig_pred_qq_residuals.png}{0.80}{Prediction diagnostics: Q--Q plot of residuals.}
\optionalfigure{figures/fig_pred_scale_location.png}{0.90}{Prediction diagnostics: scale-location plot.}
\optionalfigure{figures/fig_pred_residual_acf_pacf_ljungbox.png}{1.00}{Prediction diagnostics: residual ACF/PACF and Ljung--Box p-values.}
\optionalfigure{figures/fig_pred_rolling_rmse_mae.png}{0.95}{Prediction diagnostics: rolling RMSE and MAE over time.}
\optionalfigure{figures/fig_pred_coefficient_paths.png}{0.95}{Prediction diagnostics: rolling coefficient paths with confidence ribbons.}
\optionalfigure{figures/fig_pred_partial_regression_grid.png}{1.00}{Prediction diagnostics: partial regression plots for key regressors.}
\optionalfigure{figures/fig_pred_cooks_distance.png}{0.95}{Prediction diagnostics: Cook's distance by observation year.}
\optionalfigure{figures/fig_pred_leverage_vs_resid2.png}{0.90}{Prediction diagnostics: leverage vs residuals with Cook's contours.}

\section{Main Tables}
\optionaltableinput{tables/implied_gatm_metrics.tex}{Implied atmospheric growth model metrics.}
\optionaltableinput{tables/table_coefficients.tex}{Estimated coefficients for sink and imbalance equations.}
\optionaltableinput{tables/table_residual_diagnostics.tex}{Residual diagnostics (serial correlation, normality, and variance checks).}
\optionaltableinput{tables/table_rcp_emissions_paths_summary.tex}{Summary of implied emissions pathways across scenarios/specifications.}
\optionaltableinput{tables/table_rcp_key_year_emissions.tex}{Key-year implied emissions (2030, 2050, 2100), peak year, and net-zero year.}
\optionaltableinput{tables/table_rcp_2050_ranking.tex}{Ranking of scenarios/specifications by 2050 emissions.}
\optionaltableinput{tables/table_rcp_inversion_params.tex}{Recovered inversion parameters ($A$, $B$, $D$, denominator, and $\kappa$).}
\optionaltableinput{tables/table_pred_model_comparison.tex}{Prediction model comparison across specifications (fit and out-of-sample metrics).}
\optionaltableinput{tables/table_pred_coefficient_stability.tex}{Rolling-window coefficient stability (estimates and confidence interval widths).}
\optionaltableinput{tables/table_pred_robustness.tex}{Robustness checks with HC3, HAC, and alternative lag structure.}
\optionaltableinput{tables/table_pred_collinearity_vif.tex}{Collinearity diagnostics: variance inflation factors.}
\optionaltableinput{tables/table_pred_collinearity_top_corr.tex}{Collinearity diagnostics: top absolute regressor correlations.}
\optionaltableinput{tables/table_pred_residual_diagnostics.tex}{Residual diagnostics with pass/fail flags at 5\% significance.}
\optionaltableinput{tables/table_pred_influence_top10.tex}{Top 10 influential observations by Cook's distance/leverage/DFFITS.}
\optionaltableinput{tables/table_pred_rolling_error_summary.tex}{Rolling forecast error summary statistics.}

\section{Interpretation Guidance}
\begin{itemize}
    \item Compare linear vs saturation curves to quantify additional mitigation pressure under nonlinear sink limits.
    \item Use key-year tables to report policy-relevant checkpoints and net-zero timing differences.
    \item Use the scenario envelope figure to communicate uncertainty bands induced by climate-state assumptions.
    \item Cross-check diagnostics tables before asserting structural conclusions from coefficients.
\end{itemize}

\section{Limitations and Next Steps}
\begin{itemize}
    \item Results depend on driver specification and estimation window choice.
    \item RCP inversion quality is sensitive to concentration file integrity and conversion assumptions.
    \item Future extension: explicit uncertainty propagation (parameter/posterior sampling) into RCP envelopes.
\end{itemize}

\end{document}
